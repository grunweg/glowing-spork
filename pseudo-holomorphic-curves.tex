\section{Pseudo-holomorphic curves}
% We have now entered the machine room of our proof and seen some ingredients in motion.
Let's examine our main machine in more detail. I've promised you pseudo-holomorphic curves, so here we go. What is a pseudo-holomorphic curve?
\begin{frame}
  \frametitle{Pseudo-holomorphic curves}
\end{frame}

\begin{definition}
  An \highlight{almost complex structure} on a smooth manifold $M$ is a collection of maps $J_p\colon T_pM\to T_pM$ with $J_p^2=-\id$, smoothly varying in $p$.
\end{definition}
\begin{theorem}
  Every symplectic manifold admits an almost complex structure.
\end{theorem}

% An almost complex manifold is a smooth manifold with a choice of almost complex structure. A pseudo-holomorphic curve is a map between two almost complex manifolds - that explains the word "ps-holo". It's domain is one-dimensional; we already saw them today.
\begin{definition}
  A \highlight{Riemann surface} is a smooth surface with a choice of almost complex structure.
\end{definition}
\begin{definition}
  A \highlight{pseudo-holomorphic curve} is a smooth map $u\colon (\Sigma,j)\to (M,J)$ with $J\circ du=du\circ j$,
where $(\Sigma,j)$ is a closed Riemann surface and $(M,J)$ an almost complex manifold.
\end{definition}


% In case you get confused: a pseudo-holomorphic curve is (the image of)a two-dimensional real manifold --- but as a complex manifold, it's one-dimensional (hence called a curve).
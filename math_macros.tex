%%%%%%%%%%%%%%%%%%%%%%%%%%%%%%%%%
% MATH OPERATORS
%%%%%%%%%%%%%%%%%%%%%%%%%%%%%%%%%
% Declaring some math operators which I needed at some point.
\DeclareMathOperator{\aut}{Aut} % automorphism group of a curve or surface
\DeclareMathOperator{\Aut}{Aut}
\DeclareMathOperator{\Bil}{Bil} % set of bilinear forms
\DeclareMathOperator{\Crit}{Crit} % set of critical points
\DeclareMathOperator{\Diff}{Diff} % diffeomorphism group of a smooth manifold
\DeclareMathOperator{\End}{End} % endomorphism bundle
\DeclareMathOperator{\Fix}{Fix} % fixed point set
\DeclareMathOperator{\support}{supp} % support of a (cont. or smooth) function
\DeclareMathOperator{\Gal}{Gal} % Galois group of a field extension
\DeclareMathOperator{\GL}{GL} % general linear group of a vector space
\DeclareMathOperator{\Hess}{Hess} % Hessian
\DeclareMathOperator{\Hom}{Hom} % space of homorphisms/homomorphism bundle
\DeclareMathOperator{\Mat}{Mat} % space of matrices
\DeclareMathOperator{\Mod}{Mod} % category of $R$-modules
\DeclareMathOperator{\Mor}{Mor} % morphisms in a general category
\DeclareMathOperator{\Ob}{Ob} % objects in a category
\DeclareMathOperator{\PSL}{PSL} % projective special linear group
\DeclareMathOperator{\Proj}{Proj} % homogeneous prime ideals
\DeclareMathOperator{\Sing}{Sing} % singular set of an algebraic variety
\DeclareMathOperator{\Sp}{Sp} % symplectic group
\DeclareMathOperator{\SL}{SL} % special linear group
\DeclareMathOperator{\Spec}{Spec} % action spectrum of a contact manifold
\DeclareMathOperator{\Stab}{Stab} % stabiliser subgroup of an element
\DeclareMathOperator{\Symp}{Symp} % symplectomorphism group of a symplectic manifold
\DeclareMathOperator{\Tor}{Tor} % Tor functor (for cohomology groups)
\DeclareMathOperator{\Tr}{Tr} % trace of a map
\DeclareMathOperator{\codim}{codim} % codimension of a set
\DeclareMathOperator{\coker}{coker} % cokernel of a map
\DeclareMathOperator{\depth}{depth} % depth of a module
\DeclareMathOperator{\diam}{diam} % diameter of a metric space
\DeclareMathOperator{\dist}{dist}
\DeclareMathOperator{\graph}{graph} % graph of a function
\DeclareMathOperator{\ggT}{ggT}
\DeclareMathOperator{\id}{id} % identity element or matrix
\DeclareMathOperator{\im}{im} % image of a map (amsmath provides ker already).
\DeclareMathOperator{\ind}{ind} % (Morse) index
%\DeclareMathOperator{\interior}{Int} % interior of a set; "\int" is already taken
\newcommand{\interior}[1]{\mathring{#1}}
\DeclareMathOperator{\lcm}{lcm} % least common multiple
\DeclareMathOperator{\md}{md} % minimal discrepancy of a singularity
\DeclareMathOperator{\ord}{ord} % order of a group element
\DeclareMathOperator{\rk}{rk} % rank of a matrix, linear map or quadratic form
\DeclareMathOperator{\sgn}{sgn} % sign/sign function
\DeclareMathOperator{\sign}{sign} % signature of a quadratic form
\DeclareMathOperator{\vol}{vol} % volume of a manifold
\DeclareMathOperator{\Sections}{\Gamma} % space of (smooth) sections of a vector or fiber bundle

%%%%%%%%%%%%
% DELIMITERS
%%%%%%%%%%%%
\usepackage{mathtools}
% according to https://tex.stackexchange.com/questions/498/mid-vertical-bar-vert-lvert-rvert-divides,
% | is a symbol (treated without spacing); \vert produces the same symbol and should be used for operators on ordinals
% (but not as a delimiter); whereas \lvert and \rvert produce left and right delimiters
\DeclarePairedDelimiter{\abs}{\lvert}{\rvert}
\newcommand{\card}[1]{\lvert#1\rvert}
\newcommand{\pairing}[2]{\langle #1, #2 \rangle}
\newcommand{\generator}[1]{\langle #1 \rangle}
% My math relations stuff.
\DeclarePairedDelimiter{\ceil}{\lceil}{\rceil}
\DeclarePairedDelimiter{\floor}{\lfloor}{\rfloor}
\DeclarePairedDelimiter{\scalarProduct}{\langle}{\rangle}
\DeclarePairedDelimiter{\norm}{\lVert}{\rVert}

%%%%%%%%%%%%%%%%%%%%%%%%%
% SHORTHAND COMMANDS.
% No special purpose, just as shorthands for my own mathematical writing.
%%%%%%%%%%%%%%%%%%%%%%%%%
\newcommand{\disc}{\mathbb{D}} % closed unit disc
\newcommand{\complexPlane}{\mathbb{C}} % complex plane/ set of complex numbers
\newcommand{\complex}{\mathbb{C}} % set of complex numbers
\newcommand{\real}{\mathbb{R}} % the set of real numbers
\newcommand{\realPlane}{\mathbb{R}^2} % real plane
\newcommand{\RP}[1]{\mathbb{RP}^{#1}} % real projective n-space
\newcommand{\CP}[1]{\mathbb{CP}^{#1}} % complex projective n-space
\newcommand{\sphere}[1]{\mathbb{S}^{#1}} % n-sphere

\newcommand{\intersect}{\cap} % set intersection, small size
\newcommand{\Intersect}{\bigcap} % set intersection, large size
\newcommand{\union}{\cup} % set union, small size
\newcommand{\Union}{\bigcup} % set union, large size
\newcommand{\disjunion}{\sqcup} % disjoint union, with a square symbol
\newcommand{\absCont}[2]{#1\ll #2} % denotes #1 being absolutely continuous w.r.t. #2
\newcommand{\divides}{\mid}
\newcommand{\notdivides}{\nmid}
% Some group theory stuff.
\newcommand{\subgp}{\leqslant}
\newcommand{\nsubgp} {\trianglelefteq}
\newcommand{\trivgp}{\langle e \rangle} % trivial group
\newcommand{\dirlim}{\varinjlim} % direct limit
\newcommand{\boundary}{\partial} % boundary of a subspace
\newcommand{\del}{\partial}
% Useful commands for writing algebra.
\newcommand{\tensor}{\otimes}
\newcommand{\isom}{\cong}
\newcommand{\dsum}{\oplus} % direct sum
\newcommand{\Dsum}{\bigoplus} % direct sum, large symbol
\newcommand{\restricted}[1]{\vert_{#1}}
\newcommand{\quotient}{/{\sim}}
\newcommand{\closure}[1]{\overline{#1}} % (topological) closure of a set
% This one was useful when doing Galois theory.
\newcommand{\algclosure}[1]{\overline{#1}}
\newcommand{\suchthat}{\,\mid\;} % best notation depends on the context!
% Make some things easier to type.
\newcommand{\std}{\text{std}}
\newcommand{\loc}{\text{loc}}
\newcommand{\reg}{\text{reg}}
\newcommand{\transverse}{\pitchfork} % transverse intersection

% Define some letters in blackboard bold or calligraphic font. This might be bad practice.
% should be \upperHalfPlane instead: \newcommand{\HH}{\mathbb{H}}
\newcommand{\N}{\mathbb{N}}
\newcommand{\PP}{\mathbb{P}}
\newcommand{\Q}{\mathbb{Q}}
\newcommand{\R}{\mathbb{R}}
\newcommand{\T}{\mathbb{T}}
\newcommand{\Z}{\mathbb{Z}}
\newcommand{\LL}{\mathcal{L}}
\newcommand{\GG}{\mathcal{G}}
\newcommand{\OO}{\mathcal{O}}
\newcommand{\FF}{\mathcal{F}}
% My own commands for convenience.
\newcommand{\st}{\colon}
\newcommand{\ds}{\, \mathrm{d}s}
\newcommand{\dt}{\, \mathrm{d}t}
\newcommand{\dx}{\, \mathrm{d}x}
\newcommand{\dy}{\, \mathrm{d}y}
\newcommand{\dz}{\, \mathrm{d}z}

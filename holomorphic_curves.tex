\section{Pseudo-holomorphic curves}
% We have now entered the machine room of our proof and seen some ingredients in motion. Let's examine our main machine in more detail. I've promised you pseudo-holomorphic curves, so here we go. What is a pseudo-holomorphic curve?
% Hearing "holomorphic", we'd expect something whose differential is complex linear, so we should talk about what complex multiplication means.
\begin{frame}
  \frametitle{Pseudo-holomorphic curves}
  \begin{definition}
    An \highlight{almost complex structure} on a smooth manifold $M$ is a collection of maps $J_p\colon T_pM\to T_pM$ with $J_p^2=-\id$, smoothly varying in $p$.
  \end{definition}
  % morally, an almost complex structure corresponds to multiplication by i on each tangent space (and indeed, each tangent space becomes a complex vector space this way)
  \begin{theorem}
    Every symplectic manifold admits an almost complex structure.
  \end{theorem}
  % In fact, there are many possible choices of almost complex structure. You should regard the choice of acs as an auxiliary object; the particular choice is not important. For instance, the space of all compatible acs is contractible - which implies many constructions are independent of the *choice* of acs.
\end{frame}

\begin{frame}
  \frametitle{Pseudo-holomorphic curves}
    % An almost complex manifold is a smooth manifold with a choice of almost complex structure. A pseudo-holomorphic curve is a map between two almost complex manifolds - that explains the word "ps-holo". It's domain is one-dimensional; we already saw them today.
  \begin{definition}
    A \highlight{Riemann surface} is a smooth surface with a choice of almost complex structure.
  \end{definition}
  % Today, we are interested in closed Riemann surfaces: compact without boundary. In fact, we have already seen all of them. If Sigma is a Riemann surface, it is diffeomorphic (and even biholomorphic) to the standard genus g surface, for some g\geq 0. We call g the genus of $\Sigma$
  % One can talk about punctured Riemann surfaces (not today; Floer cylinders are punctured holo curve). I don't think surfaces of infinite type are ever studied...
  \begin{fact}
    If $(\Sigma,j)$ is a Riemann surface and $\Sigma$ is closed, then $(\Sigma,j)\cong (\Sigma_g,j')$ for some $g\geq 0$. We call $g$ the \highlight{genus} of $\Sigma$.
  \end{fact}
  \begin{definition}
    A \shrink{closed} \highlight{pseudo-holomorphic curve} is a smooth map $u\colon (\Sigma,j)\to (M,J)$ with $J\circ du=du\circ j$,
    where $(\Sigma,j)$ is a closed Riemann surface and $(M,J)$ an almost complex manifold.
  \end{definition}
  % In case you get confused: a pseudo-holomorphic curve is (the image of)a two-dimensional real manifold --- but as a complex manifold, it's one-dimensional (hence called a curve).
\end{frame}

\begin{frame}
  \frametitle{Moduli space of holomorphic curves}
  % one key idea is to consider the moduli space of all holomorphic curves with given data
  % We've already seen that we can group them by their genus. We can also group by their homology class, the pushforward of the domain's fundamental class under the curve

  given: $(M,\omega)$ symplectic, almost complex structure $J$ on $M$\\
  for $g\geq 0$ and $A\in H_2(M)$, consider the \highlight{moduli space} of holomorphic curves
  \[ \mathcal{M}_g(A,J) := \{ u\colon (\Sigma,j)\to (M,J)\;\mid\; \text{u ps.-holo}; \Sigma\cong\Sigma_g, u_*[\Sigma]=A \}/_\sim \]
  % we quotient by reparametrisation (i.e. relating by a biholomorphic map on Sigma): fundamentally, in applications we care about the image of the curve in M

  intuition: $J$ is an auxiliary object
  % In our dream world, every moduli space is a compact smooth manifold. In fact, there is a formula for its dimension in terms of A and g.
  \begin{block}{Wishful thinking}
    $\mathcal{M}_g(A,J)$ is a compact smooth manifold (and finite-dimensional).
  \end{block}
\end{frame}

\begin{frame}
  \frametitle{Understanding the moduli space of holomorphic curves}
  \begin{block}{Wishful thinking}
    \fade{$\mathcal{M}_g(A,J)$ is a compact smooth manifold (and finite-dimensional).}
  \end{block}
  % How would we approach proving this? We can use the implicit function theorem. Let's rephrase the equation defining a holomorphic curve: multiply both sides by J and use J²=-id.
  \begin{itemize}
    \item rephrase: $u\colon (\Sigma,j)\to (M,J)$ is $J$-holomorphic iff $J\circ du\circ j=-du$ iff $du+J\circ du\circ j=0$
    % in other words, the moduli space is the zero set of the map sending u and J to du plus J composed with du composed with j
    \item so: $\mathcal{M}_g(A,J)$ is the zero set of $\Phi\colon(u,J)\mapsto du+J\circ du\circ j$
  \end{itemize}
  \pause
  % This is the kind of statement coming from the implicit function theorem.
  \begin{block}{\shrink{Finite-dimensional} Implicit function theorem}
    $E\to B$ smooth vector bundle, $s\colon B\to E$ smooth section transverse to the zero section. Then $s^{-1}(0)\subset B$ is a smooth submanifold.
  \end{block}
  % Looking closely: the map above lives in a vector bundle --- its base is the product of all smooth maps from Sigma to M with the space of all acs. This is infinite-dimensional (as is its rank).
  domain of $\Phi$ is $C^\infty(\Sigma,M)\times\mathcal{J}(M,\omega)$,
  where $\mathcal{J}(M,\omega)$ is the space of all \shrink{compatible} almost complex structures
\end{frame}

% The implicit function also holds in infinite dimensions, but with extra hypotheses. This adds further complications.
\begin{frame}
  \frametitle{Infinite-dimensional complications}
  \fade{$\mathcal{M}_g(A,J)$ is the zero set of $\Phi\colon C^\infty(\Sigma,M)\times\mathcal{J}(M,\omega)\to \dots$, $(u,J)\mapsto du+J\circ du\circ j$}
  \begin{itemize}
    % One additional condition in infinite dimensionals issue is that the linearisation of the section has a bounded right inverse. (This is automatic in finite dimensions.) This follows as the linearisation of Phi will be a Fredholm operator.
    \item linearisation of section has a bounded inverse:\\ok, $d\Phi$ is a \highlight{Fredholm operator}\pause

    % perhaps the biggest one: the domain of a Phi is not complete; the space of smooth maps is not complete - hence we cannot apply the IFT
    \item domain must be a \highlight{Banach manifold}:\\
    but $C^\infty(\Sigma,M)$ is not complete!
    % The solution is to extend Phi to a larger domain which is complete. Most commonly, we extend to a suitable Sobolev space of maps.
    % The condition $kp>2$ is necessary so the Sobolev embedding theorem guarantees continuity of our Sobolev maps.
    \item solution: extend $\Phi$ to a larger domain,\\
    e.g.\ \highlight{Sobolev spaces} $W^{k,p}(\Sigma,M)$ for $kp>2$
    % A priori, the zero set of this extended Phi could be larger: in our case, we can exclude this. The Cauchy-Riemann equation is an elliptic partial differential equation. Hence we can apply regularity results for elliptic operators to show that every solution of the C-R eqn is in fact smooth.
    \item \highlight{elliptic regularity}: extension has same zero set
  \end{itemize}
  % Note two things I have not talked about: transversality of the section and compactness of the moduli space. That comes now.
\end{frame}

\begin{frame}
  \frametitle{Bad news: transversality and compactness}
  % Our wishful thinking was too optimistic for two further reasons.
  \begin{itemize}
    % firstly, the moduli space is not compact: but it can be compactified. We just need to add some condition on our almost compatible structure. For instance, it suffices to assume the the acs is compatible, i.e. plugging it into the symplectic form induces a Riemannian metric.
    \item $\mathcal{M}_g(A,J)$ is not compact, but compactifiable:\\
    require compatible $J$ (i.e. $\omega(\cdot,J\cdot)$ Riemannian metric)
    % secondly, transversality is not always true: there are examples of J such that M is not a manifold. The best we can hope for is that almost any J satifies this. (For the experts: the set of such J is comeagre.)
    \item transversality failure: for some $J$, $\mathcal{M}_g(A,J)$ is not a manifold\\
    best case: holds for ``generic'' $J$
    % In fact, even that is too much to ask: as a rule of thumb, transversality results are incompatible with symmetries.
    % One instance of this are multiply covered curves: a holomorphic curve is multiply covered iff it factors through a (possibly branched) cover of Riemann surfaces. The cover's deck transformations act on the curve, obstructing transversality.
    % (Of course, this also applie if an external group acts on M.)
    \item more generally: transversality doesn't like symmetry\\
    e.g. multiply covered curves (or external group action)
  \end{itemize}
% To summarise: the space of all simple curves (not multiply covered) is a smooth manifold, for generic compatible $J$.
\begin{theorem}
  For ``almost all'' compatible $J$, $\mathcal{M}^*_g(A,J)$ is a smooth compactifiable manifold\shrink{ of dimension $(n-3)(2-2g)+2\langle c_1(TM), A\rangle$}.
\end{theorem}
\end{frame}
